\documentclass[12pt,twoside]{report}

%%%%%%%%%%%%%%%%%%%%%%%%%%%%%%%%%%%%%%%%%%%%%%%%%%%%%%%%%%%%%%%%%%%%%%%%%%%%%
\newcommand\fnote[1]{\captionsetup{font=scriptsize, justification=raggedright, singlelinecheck=false}\subcaption*{\textit{#1}}}

%%%%%%%%%%%%%%%%%%%%%%%%%%%%%%%%%%%%%%%%%%%%%%%%%%%%%%%%%%%%%%%%%%%%%%%%%%%%%

\newcommand{\reporttitle}{Title}
\newcommand{\reportauthor}{Michael David Hollins}
\newcommand{\supervisor}{Dr Ovidiu Serban}
\newcommand{\degreetype}{MSc AI}

%%%%%%%%%%%%%%%%%%%%%%%%%%%%%%%%%%%%%%%%%%%%%%%%%%%%%%%%%%%%%%%%%%%%%%%%%%%%%

% load some definitions and default packages
\input{includes}
\input{notation}
\date{September 2024}

\begin{document}

% load title page
\input{titlepage}


% page numbering etc.
\pagenumbering{roman}
\clearpage{\pagestyle{empty}\cleardoublepage}
\setcounter{page}{1}
\pagestyle{fancy}

%%%%%%%%%%%%%%%%%%%%%%%%%%%%%%%%%%%%
\begin{abstract}
Your abstract.
\end{abstract}

\cleardoublepage
%%%%%%%%%%%%%%%%%%%%%%%%%%%%%%%%%%%%
\section*{Acknowledgments}
Comment this out if not needed.

\clearpage{\pagestyle{empty}\cleardoublepage}

%%%%%%%%%%%%%%%%%%%%%%%%%%%%%%%%%%%%
%--- table of contents
\fancyhead[RE,LO]{\sffamily {Table of Contents}}
\tableofcontents 


\clearpage{\pagestyle{empty}\cleardoublepage}
\pagenumbering{arabic}
\setcounter{page}{1}
\fancyhead[LE,RO]{\slshape \rightmark}
\fancyhead[LO,RE]{\slshape \leftmark}

%%%%%%%%%%%%%%%%%%%%%%%%%%%%%%%%%%%%
\chapter{Introduction}

\begin{figure}[tb]
\centering
\includegraphics[width = 0.4\hsize]{./figures/imperial}
\caption{Imperial College Logo. It's nice blue, and the font is quite stylish. But you can choose a different one if you don't like it.}
\label{fig:logo}
\end{figure}

Figure~\ref{fig:logo} is an example of a figure. 

%%%%%%%%%%%%%%%%%%%%%%%%%%%%%%%%%%%%
\chapter{Background}

This chapter introduces the background to Scope 3 greenhouse gas (GHG) emissions reporting, standards and data. We begin with the historical and policy context in Sections \ref{sec:HistoricalContext} and \ref{sec:PolicyContext} respectively. This serves as a springboard to consider how firms are expected to report their Scope 3 emissions in Section \ref{sec:Scope3Reporting}. Next, we cover the data quality issues inherent to this reporting process in Section \ref{sec:Scope3Challenges}. Finally, we conclude by reviewing various approaches in the literature to tackling these issues for Scope 3 modelling in Section \ref{sec:Scope3Modelling}. 

\section{Historical context}\label{sec:HistoricalContext}
Beginning in the 1800s, naturalists were puzzled to find signs of glacial activity in places too warm for glaciers in modern times \cite{young1995}. As the century progessed, some scientists discovered that certain gases such as carbon dioxide caused a so-called ``greenhouse effect'' in their experiments \cite{Ekholm1901}; therefore, they argued that these greenhouse gases (GHGs) might affect the planet's temperature through trapping heat in the atmosphere \cite{Arrhenius1896}, and so past climates could plausibly have been very different from our own \cite{foote1856}. Consistent with this theory, during the twentieth century, scientists observed both rising global surface temperatures and steep increases in concentrations of GHGs \cite{Sawyer1972}. Accordingly, today's mainstream scientific consensus is that the earth's recent climate change is mostly driven by human activity: as economic development has increased the combustion of fossil fuels, this has raised GHG levels in the atmosphere and thereby global temperatures \cite{IPCC2021, RS2020}. 

\begin{figure}[H]
\centering
\caption{Global average temperature anomalies and greenhouse gas emissions}
	\begin{subfigure}[t]{0.475\textwidth}
		\centering
		\caption{Average global temperature anomaly}
		\includegraphics[width=\linewidth]{world\_temp\_anomalies.png}
		\fnote{Source: \href{https://ourworldindata.org/co2-and-greenhouse-gas-emissions}{Our World in Data \cite{Ritchie2023}}. Global average land-sea temperature anomaly relative to the 1961-1990 average temperature, in degrees Celsius.}
		\label{fig:WorldTempAnomalies}
	\end{subfigure}
	~~
	\begin{subfigure}[t]{0.475\textwidth}
		\centering
		\caption{Global greenhouse gas emissions}
		\includegraphics[width=\linewidth]{world\_ghg\_emissions.png}
		\fnote{Source: \href{https://ourworldindata.org/co2-and-greenhouse-gas-emissions}{Our World in Data \cite{Ritchie2023}}. Greenhouse gas emissions include carbon dioxide, methane and nitrous oxide from all sources, including land-use change. They are measured in billions of tonnes of carbon dioxide-equivalents over a 100-year timescale.}
		\label{fig:WorldGHGEmissions2}
	\end{subfigure}
\end{figure}

Accompanying the growing weight of scientific evidence was political conviction that something must be done. Consequently, landmark international treaties such as the 1997 Kyoto Protocol \cite{UN1997} and the 2015 Paris Agreement \cite{UNFCCC2020} legally mandated that countries reduce their GHG emissions. Adding to the scientific and political momentum around climate change, growing public concern for the environment further catalysed domestic support for national targets such as Net Zero \cite{Poortinga2023}. 

\section{Policy context}\label{sec:PolicyContext}
To meet GHG reduction targets, GHG emissions must be accurately and consistently measured. Therefore, as policy makers recognised the need for an international standard for corporate GHG accounting and reporting, in 1998, the \href{https://ghgprotocol.org/about-us}{GHG Protocol Initiative} was launched with the mission to develop GHG reporting standards for businesses \cite{ghgprotocol2004}. Since then, the GHG Protocol has increasingly been incorporated into global reporting standards including the Global Reporting Initiative (GRI 305, \cite{gri2016}), the International Financial Reporting Standards Foundation (IFRS S2, \cite{ifrs2023}) and the Taskforce on Climate-related Financial Disclosures (TCFD, \cite{tcfd2021}).
\\ \\
Simply put, the GHG Protocol provides a framework for firms to account for their GHG emissions, which are first divided into \textit{direct} and \textit{indirect}. Direct emissions come from sources that are owned or controlled by the company. On the other hand, while indirect emissions are a consequence of activities of the company, they occur at sources owned or controlled outside the company. 
\\ \\
To help delineate this further, three ``scopes'' are defined for GHG accounting and reporting purposes. First,  \textbf{Scope 1} emissions are \textbf{direct} GHG emissions from sources which are owned or controlled by the company. Second, \textbf{Scope 2} emissions are the \textbf{indirect} GHG emissions from the generation of energy that is purchased, including electricity, steam, heat and cooling. Finally, \textbf{Scope 3} comprises \textbf{all other indirect emissions} that occur as a consequence of the company's activities but from sources not owned or controlled by the company. Together, these scopes encompass the total emissions footprint of a company as summarised in Figure \ref{fig:Scope3Diagram} below.

\begin{figure}[H]
	\centering
	\caption{Overview of GHG Protocol scopes and emissions across the value chain}
	\label{fig:Scope3Diagram}
	\includegraphics[width=\linewidth]{scope\_3\_diagram.png}
	\fnote{Source: \cite{ghgscope32013}, p.6}
\end{figure}

In accordance with \textit{The Greenhouse Gas Protocol Corporate Accounting and Reporting Standard}, many jurisdictions require mandatory Scope 1 and Scope 2 disclosures for publicly listed firms, including in the EU \cite{eu20232772}, the US \cite{sec2024}, the UK \cite{ukleg2018} and Japan \cite{fsa2022}. Furthermore, although at the moment Scope 3 reporting is optional in these key financial markets, the regulatory landscape in all of them is shifting away from voluntary efforts and towards mandatory Scope 3 disclosures \cite{ftserussell2024}. 
\\ \\
Considering the UK's policy in particular, the UK Government's stated ambition is to introduce mandatory Scope 3 reporting across the entire economy by 2025 in line with the recommendations of the TCFD \cite{ukgov2020}. This would require each reporting entity to ``comply or explain'' their Scope 3 emissions. In other words, if deemed material, each entity must report their Scope 3 GHG emissions in line with the GHG Protocol format. Meanwhile, for listed companies, since 2020 the Financial Conduct Authority (FCA) has phased in mandatory Scope 3 disclosures \cite{fca2020, fca2021}. 
\\ \\
\section{Scope 3 reporting}\label{sec:Scope3Reporting}
Having set out the origins and development of GHG emission targets, it remains to consider what is required for a firm to comply with the GHG Protocol \textit{Corporate Value Chain (Scope 3) Accounting and Reporting Standard} \cite{ghgscope32013}. A detailed understanding of this process will help illuminate the issues with data quality which are covered in Section \ref{sec:Scope3Challenges}.
\\ \\
After the firm has divided its GHG emissions into direct and indirect, the first task is to \textbf{set the Scope 3 boundary}; in short, this means accounting for all Scope 3 emissions in the value chain and to disclose and justify any exclusions. This includes the gases contained in the Kyoto Protocol: carbon dioxide (C0\textsubscript{2}), methane (CH\textsubscript{4}), nitrous oxide (N\textsubscript{2}0), hydrofluorocarbons (HFCs), perfluorocarbons (PFCs) and lastly sulphur hexaflouride (SF\textsubscript{6}). Furthermore, the company is required to account for these emissions in their respective Scope 3 categories as listed in Table \ref{tab:Scope3Categories}. 

\begin{table}[H]
\caption{Scope 3 Categories}
\label{tab:Scope3Categories}
\resizebox{\textwidth}{!}{%
\begin{tabular}{ccll}
\multicolumn{1}{l}{} &
  \textbf{\#} &
  \multicolumn{1}{c}{\textbf{Category}} &
  \multicolumn{1}{c}{\textbf{Description}} \\ \hline
 \parbox[t]{5mm}{\multirow{8}{*}{\rotatebox[origin=c]{90}{Upstream}}} &
  1 &
  Purchased goods and services &
  \begin{tabular}[c]{@{}l@{}}Extraction, production, and transportation of goods and services \\ purchased or acquired by the reporting company\end{tabular} \\ \cline{2-4} 
 &
  2 &
  Capital goods &
  \begin{tabular}[c]{@{}l@{}}Extraction, production, and transportation of capital goods purchased\\  or acquired by the reporting company\end{tabular} \\ \cline{2-4} 
 &
  3 &
  \begin{tabular}[c]{@{}l@{}}Fuel- and energy-related activities \\ (not included in Scopes 1 or 2)\end{tabular} &
  \begin{tabular}[c]{@{}l@{}}Extraction, production, and transportation of fuels and energy \\ purchased or acquired by the reporting company not already \\ accounted for in Scope 1 or Scope 2\end{tabular} \\ \cline{2-4} 
 &
  4 &
  Upstream transportation and distribution &
  \begin{tabular}[c]{@{}l@{}}Transportation and distribution of products or services purchased by the\\  reporting company between their tier 1 suppliers and own operations,\\  including inbound and outbound logistics and inter-facility transport\end{tabular} \\ \cline{2-4} 
 &
  5 &
  Waste generated in operations &
  \begin{tabular}[c]{@{}l@{}}Disposal and treatment of waste generated in the reporting \\ company's operations\end{tabular} \\ \cline{2-4} 
 &
  6 &
  Business travel &
  Transportation of employees for business-related activities \\ \cline{2-4} 
 &
  7 &
  Employee commuting &
  Transportation of employees between their homes and their workplaces \\ \cline{2-4} 
 &
  8 &
  Upstream leased assets &
  \begin{tabular}[c]{@{}l@{}}Operation of assets leased by the reporting company not included \\ in Scope 1 or Scope 2\end{tabular} \\ \hline
\parbox[t]{5mm}{\multirow{7}{*}{\rotatebox[origin=c]{90}{Downstream}}} &
  9 &
  Downstream transportation and distribution &
  \begin{tabular}[c]{@{}l@{}}Transportation and distribution of products sold by the company between\\ the company's operations and the end consumer, including retail and storage\end{tabular} \\ \cline{2-4} 
 &
  10 &
  Processing of sold products &
  Processing of intermediate products sold by downstream companies \\ \cline{2-4} 
 &
  11 &
  Use of sold products &
  End use of goods and services sold by the reporting company \\ \cline{2-4} 
 &
  12 &
  End-of-life treatment of sold products &
  \begin{tabular}[c]{@{}l@{}}Waste disposal and treatment of products sold by the reporting company \\ at the end of their life\end{tabular} \\ \cline{2-4} 
 &
  13 &
  Downstream leased assets &
  \begin{tabular}[c]{@{}l@{}}Operation of assets owned by the reporting company and leased to other \\ entities not included in Scope 1 and Scope 2\end{tabular} \\ \cline{2-4} 
 &
  14 &
  Franchises &
  Operation of franchises not included in Scope 1 and Scope 2 \\ \cline{2-4} 
 &
  15 &
  Investments &
  \begin{tabular}[c]{@{}l@{}}Operation of investments (including equity and debt investments and \\ project finance) not included in Scope 1 or Scope 2\end{tabular} \\ \hline
\end{tabular}%
}
\fnote{Source: Adapted from The GHG Protocol \textit{Corporate Value Chain (Scope 3) Accounting and Reporting Standard} \cite{ghgscope32013}}
\end{table}

These distinct categories are designed to help firms organise, understand and report on the diversity of Scope 3 activities throughout their value chain. However, while aiming for completeness, it is clear that in some cases, accounting for all of these emissions will not be feasible.  Therefore, the GHG Protocol concedes that while some omissions are permitted, ``Companies should ensure that the scope 3 inventory appropriately reflects the GHG emissions of the company, and serves the decision-making needs of users, both internal and external to the company. In particular, companies should not exclude any activity that is expected to contribute significantly to the company's total scope 3 emissions'' (\cite{ghgscope32013}, p.60).  
\\ \\
After identifying the most likely sources of significant emissions in the value chain, the firm must then \textbf{collect data} for quantifying emissions in the Scope 3 categories. There are two main methods to do this. First, where the firm has access to direct emissions data, then they can simply use the direct measurement. In practice, the second method of calculation is much more common, in which the firm uses activity data and emission factors to estimate emissions. In any case, the actual or estimated emissions data is then converted into a carbon dioxide equivalent (C0\textsubscript{2}e) so that Scope 3 emissions can be aggregated and compared regardless of GHG type. 
\\ \\
Finally, to comply with the GHG Protocol Scope 3 Standard \cite{ghgscope32013}, the firm must \textbf{report} their Scope 3 emissions with the following requirements:
\begin{itemize}
	\item Total scope emissions reported separately by each Scope 3 category;
	\item For each Scope 3 category, total GHG emissions reported in metric tons of C0\textsubscript{2} equivalent, excluding biogenic C0\textsubscript{2} emissions and independent of any GHG trades, such as purchases, sales, or transfers of offsets or allowances;
	\item A list of Scope 3 categories and activities included in the inventory;
	\item A list of Scope 3 categories or activities excluded from the inventory, and with justification for their exclusion;
	\item Once a base year has been established: the year chosen as the Scope 3 base year; the rationale for choosing the base year; the base year's recalculation policy; Scope 3 emissions by category in the base year, consistent with the base year emissions recalculation policy; and appropriate context for any significant emissions changes that triggered base year emissions recalculations;
	\item For each Scope 3 category, any biogenic C0\textsubscript{2} emissions reported separately; 
	\item For each Scope 3 category, a description of the various types and sources of data, including activity data, emission factors and global warming potential (GWP) values, used to calculate emissions, and a description of the data quality of reported emissions data;
	\item For each Scope 3 category, a description of the methodologies, the allocation methods, and assumptions used to calculate Scope 3 emissions;
	\item For each Scope 3 category, the percentage of emissions calculated using data obtained from suppliers or other value chain partners.
\end{itemize}
These reporting requirements are clearly not trivial. Such is the nature and extent of the requirements that some have concluded that they are practically unworkable in their current form \cite{patchell2018}. The following Section explores the common problems with Scope 3 emissions disclosures that are highlighted in the literature.

\section{Scope 3 data challenges}\label{sec:Scope3Challenges}

Early studies exploring the nascent Scope 3 assessments found that the data were largely incomplete and lacking rigour \cite{DownieStubbs2013}. Moreover, even after a decade, a report by FTSE-Russell concluded that the data were still characterised by large gaps and poor quality \cite{ftserussell2024}. This suggests that something in the data generating process (as outlined above in Section \ref{sec:Scope3Reporting}) could be driving these inadequate outcomes. This Section describes the main shortcomings of Scope 3 disclosures identified in the literature. Beginning with a review of challenges inherent in the data generating process, we subsequently highlight the problems this causes downstream data analysis. 

\subsection{Issues with the data generating process}\label{sec:DataGenIssues}

Klaaßen and Stoll \cite{klassenstoll2021} identify three causes of error in firms' GHG emissions data: reporting inconsistency, boundary incompleteness and activity exclusion. 

\subsubsection{Reporting inconsistency}

Firms may disclose their Scope 3 GHG emissions through corporate channels, such as annual reports, or to third parties such as the CDP. However, there is no need for these disclosures to be consistent with each other, as they are written for different purposes and audiences. Accordingly, in line with the expectations of stakeholder theory, studies have found that on average, GHG emissions disclosed in corporate reports are lower than those reported to the CDP \cite{depoers2016, klassenstoll2021}. As corporate reports are less prescriptive than the CDP questionnaire, it appears that firms use their them to paint a better picture of their GHG emissions to the wider public. Notwithstanding this, different figures do not imply fraudulent disclosure: it may simply reflect the choice of valid but different methodologies \cite{depoers2016}. In any case, the upshot is that even for a given firm in a given year which self-reports, there is potentially no consistent source of ground truth. 

\subsubsection{Boundary incompleteness}

For the 15 categories of Scope 3 GHG emissions, the GHG Protocol recommends a minimum boundary of what to include. Therefore, for each category companies are encouraged to choose the most appropriate calculation method depending on data quality and quantity. These calculation methods are either top-down (input-output based), bottom-up (processed based), or a hybrid of the two which begins bottom-up and then fills in the gaps top-down. Furthermore, for the most material categories, firms are encouraged to use primary data, i.e. data sourced directly from other firms. However, there is often limited data transparency across the value chain and so this option is commonly unavailable. Where secondary data such as industry reports and production volumes is also missing, boundary incompleteness follows. Some scholars such as Patchell \cite{patchell2018} argue that this is an intractable problem due to the complexity of global supply chains and the prohibitive costs involved. This might explain why despite the US Securities and Exchange Commission (SEC) previously intending to introduce compulsory Scope 3 disclosures, a move that some argued was a vital step for improving coverage \cite{lgim2023}, it has instead decided against this course of action following consultations \cite{sec2024}.

\subsubsection{Activity exclusion}

Finally, even when firms report their Scope 3 emissions, they may choose to omit certain relevant categories entirely. Despite the encouragements of standard setters to report Scope 3 emissions that are considered `material' (ISSB) \cite{ifrs2023}, `appropriate' (TCFD) \cite{tcfd2021}, or `significant' (GRI) \cite{gri2016},  firms often report what is easier to measure (e.g. business travel) rather than what is important to their carbon footprint. For example, in their 2022 report, CDP found that only 36\% of reporting firms disclosed their Purchased Goods and Services (Category 1) emissions, which are relevant for almost every sector \cite{cdp2022}. As this remains a persistent problem, FTSE Russell \cite{ftserussell2024} has recommended that companies begin by reporting the two most material categories for their sector, which on average should encompass over 80\% of their total Scope 3 emissions. 

\subsection{Consequential problems for data analysis}

Together, the three problems in Section \ref{sec:DataGenIssues} prove a major hinderance to analysis. Common issues include:
\begin{itemize}
	\item Missing data
	\item Biased data
	\item Volatile data
	\item Lack of comparability
\end{itemize}


\subsection{Missing data}
Consequently, the problem of missing data remains acute. Recent surveys have found that less than 50\% of firms reported emissions for at least one Scope 3 category, and only 20\% reported material Scope 3 emissions \cite{cdp2023, ftserussell2024}. 
\\ \\
Aside from there being no legal obligation, the obvious reason for the paucity of data comes from the nature of Scope 3 emissions itself. They are by definition emissions in the value chain beyond the firm's direct control or ownership. Moreover, the 

Of companies disclosing to CDP (formerly the Climate Disclosure Project), only 41\% reported emissions for one or more Scope 3 categories in 2022, despite those emissions being on aerage 11.4 times higher than operational (i.e Scopes 1 and 2) emissions \cite{cdp2022}. 

\subsection{Partial reporting}
- Availability bias
- Material data left out/ what is significant? 
- Report different numbers (see references in Klassen p.2)

\subsection{Volatility}
- Changes often unexplained

\subsection{Lack of comparability}
- Wide latitude in methodology
- Categories are not always obvious

\section{Attempts to model Scope 3 emissions}\label{sec:Scope3Modelling}

Data coverage and modelling/literature. 
%%%%%%%%%%%%%%%%%%%%%%%%%%%%%%%%%%%%
\chapter{Data}\label{sec:Data}


%%%%%%%%%%%%%%%%%%%%%%%%%%%%%%%%%%%%
\chapter{Experimental Results}


%%%%%%%%%%%%%%%%%%%%%%%%%%%%%%%%%%%%
\chapter{Conclusion}


%% bibliography
\bibliographystyle{plain}
\bibliography{C:/localtexmf/bibtex/bib/mybibs/thesis_references}

\end{document}
